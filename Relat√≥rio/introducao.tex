\chapter{Introdução}
\label{chap:intro}

% CADA SECÇÃO DEVE TER UM LABEL
% CADA FIGURA DEVE TER UM LABEL
% CADA TABELA DEVE TER UM LABEL

\section{Enquadramento}
\label{sec:amb}

Sistema de detecção e tracking de objectos em movimentocom base em análise de imagem

- Visão Computacional
- Identificação de objetos em imagem (2D, 3D) ou em um video sobre um ambiente de luz natural.
Desafio para os sistemas de Visão Computacional, sendo que já foram implementadas muitas abordagens para estes sistemas ao longo dos anos.
- O Tranking de objetos é tambem um desafio para estes sistemas de Visão Computacional.

Neste projeto todo o codigo é desenvolvido usando um Raspberry Pi, com capacidades limitadas, o que torna este trabalho ainda um maior defafio ao nivel da Complexidade Computacional.

- Complexidade Computacional 

Usaremos tambem uma camera comum e conectada ao Raspberry Pi para obter as imagens para analisar em tempo real. Usarei uma camera da marca Logitech que tenho em casa, conectada via USB.

Para isto iremos usar irei usar um Raspnerry Pi 4, Modulo B de 4GB RAM que tenho em casa e um cartão de memória de 64 Gb.

\section{Motivação}
\label{sec:mot}

O tema deste projeto é um tema atual e muito importante para as áreas de Visão Computacional e da Robotica. Daí ser uma grande motivação minha na escolha deste trabalho.

O reconhecimento de objetos será sempre uma area imprencindivel e de grande interesse para a robotica. Neste caso iremos reconhecer alguns objetos e rastrealos em um fluxo de video. 
Isto ainda em um ambiente de desenvolvimento com capacidades bastante limitadas, que é o Raspberry Pi, o que torna esta tarefa um desafio. 

Apesar de nós humanos, conseguirmos reconhecer múltiplos objetos em diferentes imagens, e em diferentes disposições em segundos sem qualquer dificuldade. Muitas vezes mesmo Objetos obtruidos por outros objetos. Isto para a Visão Computacional é ainda um desafio, e é muito mais complexo reconhecer um objeto em uma imagem do que pensamos à primeira vista.

- Lab 
- Prof
- Raspberry Pi

----- Old:

A motivação da escolha deste projeto como projeto deve-se ao meu interesse neste tema de projeto. Tenho também muito interesse no aprendizado das ferramentas utilizadas como por exemplo a programação em um dispositivo com capacidades reduzidas, como um \ac{Raspberry}.


\section{Objetivos}
\label{sec:obj}

Como objetivo deste trabalho temos a elaboração de um Sistema de deteção e traking de objetos em movimento com base em análise de imagem. Este sistema deve ser desenvolvido para posteriormente ser usado em um Raspberry Pi, com capacidade de processamento limitada. Devemos também usar uma webcam e ferremntas de programação como Tensorflow, Opencv e Yolo.

\section{Organização do Documento}
\label{sec:organ}
% !POR EXEMPLO!
De modo a refletir o trabalho que foi feito, este documento encontra-se estruturado da seguinte forma:
\begin{enumerate}
\item O primeiro capítulo -- \textbf{Introdução} -- apresenta o projeto, a motivação para a sua escolha, o enquadramento para o mesmo, os seus objetivos e a respetiva organização do documento.
\item O segundo capítulo -- \textbf{Tecnologias Utilizadas} -- descreve os conceitos mais importantes no âmbito deste projeto, bem como as tecnologias utilizadas durante do desenvolvimento da aplicação.
\item ...
\end{enumerate}

% 
\section{Algumas Dicas -- [RETIRAR DA VERSÃO FINAL]}
% ALGUMAS DICAS
Os relatórios de projeto são individuais e preparados em \LaTeX, seguindo o formato disponível na página da unidade curricular. Deve ser prestada especial atenção aos seguintes pontos:
\begin{enumerate}
  \item O relatório deve ter um capítulo Introdução e Conclusões e Trabalho Futuro (ou só Conclusões);
  \item A última secção do primeiro capítulo deve descrever suscintamente a organização do documento;
  \item O relatório pode ser escrito em Língua Portuguesa ou Inglesa;
  \item Todas as imagens ou tabelas devem ter legendas e ser referidas no texto (usando comando \texttt{\textbackslash ref\{\}}).
\end{enumerate}