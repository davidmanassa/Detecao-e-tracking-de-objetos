\chapter{Introdução}
\label{chap:intro}

% CADA SECÇÃO DEVE TER UM LABEL
% CADA FIGURA DEVE TER UM LABEL
% CADA TABELA DEVE TER UM LABEL

\section{Enquadramento}
\label{sec:amb}

Este relatório enquadra-se no tema de projeto \textit{Sistema de detecção e tracking de objectos em movimentocom base em análise de imagem}. Junto com este relatório irá um Sistema funcional, em que juntos, formam este Projeto.


\section{Motivação}
\label{sec:mot}

A motivação da escolha deste projeto como projeto deve-se ao meu interesse neste tema de projeto. Tenho também muito interesse no aprendizado das ferramentas utilizadas como por exemplo a programação em um dispositivo com capacidades reduzidas, como um \ac{Raspberry}.


\section{Objetivos}
\label{sec:obj}

Como objetivo deste trabalho temos a elaboração de um Sistema de deteção e traking de objetos em movimento com base em análise de imagem. Este sistema deve ser desenvolvido para posteriormente ser usado em um Raspberry Pi, com capacidade de processamento limitada. Devemos também usar uma webcam e ferremntas de programação como Tensorflow, Opencv e Yolo.

\section{Organização do Documento}
\label{sec:organ}
% !POR EXEMPLO!
De modo a refletir o trabalho que foi feito, este documento encontra-se estruturado da seguinte forma:
\begin{enumerate}
\item O primeiro capítulo -- \textbf{Introdução} -- apresenta o projeto, a motivação para a sua escolha, o enquadramento para o mesmo, os seus objetivos e a respetiva organização do documento.
\item O segundo capítulo -- \textbf{Tecnologias Utilizadas} -- descreve os conceitos mais importantes no âmbito deste projeto, bem como as tecnologias utilizadas durante do desenvolvimento da aplicação.
\item ...
\end{enumerate}

% 
\section{Algumas Dicas -- [RETIRAR DA VERSÃO FINAL]}
% ALGUMAS DICAS
Os relatórios de projeto são individuais e preparados em \LaTeX, seguindo o formato disponível na página da unidade curricular. Deve ser prestada especial atenção aos seguintes pontos:
\begin{enumerate}
  \item O relatório deve ter um capítulo Introdução e Conclusões e Trabalho Futuro (ou só Conclusões);
  \item A última secção do primeiro capítulo deve descrever suscintamente a organização do documento;
  \item O relatório pode ser escrito em Língua Portuguesa ou Inglesa;
  \item Todas as imagens ou tabelas devem ter legendas e ser referidas no texto (usando comando \texttt{\textbackslash ref\{\}}).
\end{enumerate}